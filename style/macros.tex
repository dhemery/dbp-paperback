%%%%%%%%%%%%%%%%%%%%%%%%%%%%%%%%%%%%%%%%%%%%%%%%%%%%%%%%%%%
%
% Paper and margins
%
%%%%%%%%%%%%%%%%%%%%%%%%%%%%%%%%%%%%%%%%%%%%%%%%%%%%%%%%%%%

\newdimen\pdforigintopadjustment
\newdimen\pdforiginleftadjustment
\newdimen\topadjustment
\newdimen\versoleftadjustment
\newdimen\rectoleftadjustment

% Default values
\pdforigintopadjustment=0pt
\pdforiginleftadjustment=0pt
\topadjustment=0pt
\versoleftadjustment=0pt
\rectoleftadjustment=0pt

% \setpdforigin{left adjustment}{top adjustment}
% By default (per a configuration file) the PDF renderer
% might start each page at somewhere other than the top
% left corner of the paper. To compensate for that, call
% this macro to adjust where we place the page.
\def\adjustpdforigin#1#2{\pdforiginleftadjustment=#1\pdforigintopadjustment=#2}

% \setpapersize{width}{height}
\def\setpapersize#1#2{\pdfpagewidth=#1\pdfpageheight=#2\setmargins{0pt}{0pt}{0pt}{0pt}}

% \setmargins{top}{bottom}{inner}{outer}
% Sets the margins of the page's main content. The running
% header will be placed above the main content, within the
% top margin. The running footer will be placed below the
% main content, within the bottom margin.
\def\setmargins#1#2#3#4{%
  \vsize=\pdfpageheight
  \hsize=\pdfpagewidth
  \advance\vsize by -#1
  \advance\vsize by -#2
  \advance\hsize by -#3
  \advance\hsize by -#4
  \topadjustment=#1\advance\topadjustment\pdforigintopadjustment
  \versoleftadjustment=#4\advance\versoleftadjustment\pdforiginleftadjustment
  \rectoleftadjustment=#3\advance\rectoleftadjustment\pdforiginleftadjustment
}

% Output routine to adjust inner margin depending on page number.
\def\facingpages{%
  \dimen0=\ifodd\pageno\rectoleftadjustment\else\versoleftadjustment\fi%
  \shipout\vbox{\moveright\dimen0\vbox{\vskip\topadjustment\makeheadline\pagebody\makefootline}}%
  \advancepageno%
  \ifnum\outputpenalty>-20000 \else\dosupereject\fi%
}

\def\ensureevenpagecount{\ifodd\pageno\onrectopage\else\fi}

\output{\facingpages}




%%%%%%%%%%%%%%%%%%%%%%%%%%%%%%%%%%%%%%%%%%%%%%%%%%%%%%%%%%%
%
% Pages
%
%%%%%%%%%%%%%%%%%%%%%%%%%%%%%%%%%%%%%%%%%%%%%%%%%%%%%%%%%%%

\def\onrectopage{\eject\ifodd\pageno \else\line{}\vfil\eject\fi}

\def\maketitlepage#1#2#3{% title, author, publisher
  \onrectopage
  \null\vskip1in
  \titlepagetitle{#1}
  \vfill
  \titlepageauthor{#2}
  \vfill
  \titlepagepublisher{#3}
  \vskip1in\null
  \eject
}

\def\titlepagetitle#1{{\titlepagetitlefont\raggedleft#1\par}}
\def\titlepageauthor#1{{\titlepageauthorfont\raggedleft#1\par}}
\def\titlepagepublisher#1{{\titlepagepublisherfont\raggedleft#1\par}}

% An info page is a frontmatter or backmatter page
% other than a title or halftitle page.
\def\infopagerecto#1{\onrectopage\infopage{#1}}
\def\infopage#1{\eject{\infopagetitlefont\centered\noindent#1\par}}
\def\infoheading#1{\bigskip{\infoheadingfont\centered\noindent#1\par}}
\def\infoitem#1{\smallskip{\centered\noindent #1\par}}




%%%%%%%%%%%%%%%%%%%%%%%%%%%%%%%%%%%%%%%%%%%%%%%%%%%%%%%%%%%
%
% Page Frame (Header and Footer)
%
%%%%%%%%%%%%%%%%%%%%%%%%%%%%%%%%%%%%%%%%%%%%%%%%%%%%%%%%%%%

\def\useemptypageframe{
  \global\def\makeheadline{\headlinefont\line{}}
  \global\def\makefootline{\footlinefont\line{}}
}
\def\usedisplaypageframe{
  \global\def\makeheadline{\headlinefont\line{}\usenormalpageframe}
}
\def\usenormalpageframe{
  \global\def\makeheadline{\line{\headlinefont\hfil\ifodd\pageno\title\else\author\fi\hfil}}
  \global\def\makefootline{\line{\footlinefont\hfil\folio\hfil}}
}




%%%%%%%%%%%%%%%%%%%%%%%%%%%%%%%%%%%%%%%%%%%%%%%%%%%%%%%%%%%
%
% Divisions
%
%%%%%%%%%%%%%%%%%%%%%%%%%%%%%%%%%%%%%%%%%%%%%%%%%%%%%%%%%%%

\def\frontmatter{\onrectopage\pageno=-1\useemptypageframe}
\def\mainmatter{\onrectopage\pageno=1\usenormalpageframe}
\def\backmatter{\frontmatter}

\def\chapter#1{
  \onrectopage
  \usedisplaypageframe
  \typesetchapterheading{#1}
}

\def\typesetchapterheading#1{
  \vbox{
    \vskip5\baselineskip
    {\raggedleft\chaptertitlefont#1\par}
  }
  \nobreak
}

\def\scene#1{
  \typesetsceneheading{#1}
}

\def\typesetsceneheading#1{
  \vbox{
    \vskip3\baselineskip
    {\raggedleft\scenetitlefont#1\par}
    \vskip\baselineskip
  }
  \nobreak
}

% Ignore labels added by pandoc and Scrivener
\def\label#1{}


%%%%%%%%%%%%%%%%%%%%%%%%%%%%%%%%%%%%%%%%%%%%%%%%%%%%%%%%%%%
%
% Fonts
%
%%%%%%%%%%%%%%%%%%%%%%%%%%%%%%%%%%%%%%%%%%%%%%%%%%%%%%%%%%%

\newdimen\bodyfontsize
\newdimen\leading

% \makefont{name}{system font name}{dimen}{options}
\def\makefont#1#2#3#4{\font#1="#2#4" at #3}

% \makebodyfont{name}{options}
\def\makebodyfont#1#2{\makefont{#1}{\bodyfontfamily}{\bodyfontsize}{#2}}

% \makedisplayfont{name}{dimen}{options}
\def\makedisplayfont#1#2#3{\makefont{#1}{\displayfontfamily}{#2}{#3}}




%%%%%%%%%%%%%%%%%%%%%%%%%%%%%%%%%%%%%%%%%%%%%%%%%%%%%%%%%%%
%
% Text Effects
%
%%%%%%%%%%%%%%%%%%%%%%%%%%%%%%%%%%%%%%%%%%%%%%%%%%%%%%%%%%%

\def\leadin#1{\noindent{\bodysc#1}}

% TODO: Make this handle nested emphasis.
\def\emph#1{{\bodyit #1}}

\def\raggedleft{\parindent=0pt\parfillskip=0pt\leftskip=0pt plus \hsize\relax}
\def\raggedright{\parindent=0pt\parfillskip=0pt\rightskip=0pt plus \hsize\relax}
\def\centered{\raggedleft\raggedright}

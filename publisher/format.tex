\catcode`@=11

\gdef\@defaulting#1#2{\if\relax#1\relax#2\else#1\fi}
\gdef\@optional#1#2{\if\relax#2\relax\else#1#2\fi}
\gdef\font@spec{\font@base@name\@optional{-}{\font@shape}-\font@figures\@optional{-}{\font@variant}-\font@encoding\@optional{ at }{\font@size\relax}}
\gdef\font@shape{\@defaulting{\font@weight\font@slant\font@optical@size}{\font@default@shape}}

% Declare a new font family. This specifies the terms to
% use to build file names for this font.
%
% The parameters are:
% 1 family tag
%     A tag to identify this font family.
% 2 file name prefix
%     The filename prefix that names this family's font
%     files. e.g. GaramondPremrPro
% 3 regular weight tag
%     The word that selects regular weight in this font's
%     file names. Usually empty, but may be Regular.
% 4 italic tag
%     The word that selects italic slant in this font's
%     file names. This is usually either It or Italic.
% 5 bold tag
%     The word that selects bold weight in this font's file
%     names. This is usually either Bd or Bold.
% 6 default shape tag
%     The word that selects the default font in this font's
%     file names. That is, it selects the font that is
%     regular weight, roman slant, and book or text optical
%     size. This is usually empty, but may be Regular.
\def\newfamily#1#2#3#4#5#6{%
  \expandafter\xdef\csname#1@base@name\endcsname{#2}%
  \expandafter\xdef\csname#1@regular@weight\endcsname{#3}%
  \expandafter\xdef\csname#1@italics@slant\endcsname{#4}%
  \expandafter\xdef\csname#1@bold@weight\endcsname{#5}%
  \expandafter\xdef\csname#1@default@shape\endcsname{#6}%
}

% Declare a new font style. This specifies the terms to use
% to build file names for this style.
%
% The parameters are:
% 1 style tag
%     A tag to identify this style.
% 2 family tag
%     Selects the font family to use for this tag.
% 3 figure style tag
%     The word that selects the desired figure style (the
%     kind of digits to use) in this font's file names.
%     This is one of:
%     - osf
%         Old style figures.
%         Variable height, proportional width.
%     - tosf
%         Tabular old style figures.
%         Variable height, fixed width.
%     - lf
%         Lining figures.
%         Fixed height, variable width.
%     - tlf
%         Tabular lining figures.
%         Fixed width, fixed height.
% 4 optical size tag
%     The word that selects the desired optical size in
%     this font's file names.
% 5 size
%     The point size at which to render this style. This
%     may be empty, in which case the font's default size
%     will be used.
\def\newstyle#1#2#3#4#5{%
  \expandafter\xdef\csname#1@family\endcsname{#2}%
  \expandafter\xdef\csname#1@optical@size\endcsname{#3}%
  \expandafter\xdef\csname#1@figures\endcsname{#4}%
  \expandafter\xdef\csname#1@size\endcsname{#5}%
}

\def\style#1{
  \xdef\font@family{\csname#1@family\endcsname}
  \xdef\font@base@name{\csname\font@family @base@name\endcsname}
  \xdef\font@default@shape{\csname\font@family @default@shape\endcsname}
  \xdef\font@optical@size{\csname#1@optical@size\endcsname}
  \xdef\font@figures{\csname#1@figures\endcsname}
  \xdef\font@size{\csname#1@size\endcsname}
  \rm
}

\gdef\withfontbasename#1{\xdef\font@base@name{#1}}
\gdef\withfontencoding#1{\xdef\font@encoding{#1}}
\gdef\withfontfigures#1{\xdef\font@figures{#1}}
\gdef\withfontopticalsize#1{\xdef\font@optical@size{#1}}
\gdef\withfontshape#1#2#3{\withfontweight{#1}\withfontslant{#2}\withfontopticalsize{#3}}
\gdef\withfontslant#1{\xdef\font@slant{#1}}
\gdef\withfontvariant#1{\xdef\font@variant{#1}}
\gdef\withfontweight#1{\xdef\font@weight{#1}}

\gdef\selectfont{\font\current@font=\font@spec\current@font}

\gdef\rm{\withfontweight{\csname\font@family @regular@weight\endcsname}\withfontslant{}\withfontvariant{}\withfontencoding{t1}\selectfont}
\gdef\it{\withfontslant{\csname\font@family @italics@slant\endcsname}\selectfont}
\gdef\bf{\withfontweight{\csname\font@family @bold@weight\endcsname}\selectfont}
\gdef\sc{\withfontvariant{sc}\selectfont}
\gdef\tl{\withfontvariant{titling}\withfontfigures{tlf}\selectfont}
\gdef\sym{\withfontencoding{ts1}\selectfont}

\catcode`\@=12

%%%%%%%%%%%%%%%%%%%%%%%%%%%%%%%%%%%%%%%%%%%%%%%%%%%%%%%%%%%
%
% Textblock Size and Position on the Page
%
%%%%%%%%%%%%%%%%%%%%%%%%%%%%%%%%%%%%%%%%%%%%%%%%%%%%%%%%%%%

\catcode`\@=11

\newdimen\textblock@odd@left
\newdimen\textblock@even@left

\def\setspinemargin#1{
  \textblock@odd@left=#1
  \textblock@even@left=\pdfpagewidth
    \advance\textblock@even@left by -\hsize
    \advance\textblock@even@left by -#1
}

\catcode`\@=12

\pdfhorigin=0pt
\pdfvorigin=0pt
\pdfpagewidth=5 true in
\pdfpageheight=8 true in
\hsize=22pc
\vsize=39pc
\voffset=4.5pc

\setspinemargin{4.5pc}

%%%%%%%%%%%%%%%%%%%%%%%%%%%%%%%%%%%%%%%%%%%%%%%%%%%%%%%%%%%
%
% Output Routine
%
%%%%%%%%%%%%%%%%%%%%%%%%%%%%%%%%%%%%%%%%%%%%%%%%%%%%%%%%%%%

\newif\ifclearedpage
\newif\ifdisplaypage

\catcode`@=11

% Write one page (textblock, headline, and footline).
\def\facingpages{%
  \ifclearedpage\headline={\line{}}\footline={\line{}}\fi%
  \ifdisplaypage\headline={\line{}}\footline={\displaypagefootline}\fi%
  \global\clearedpagefalse%
  \global\displaypagefalse%
  \dimen0=\ifodd\pageno\textblock@odd@left\else\textblock@even@left\fi%
  \shipout\vbox{\moveright\dimen0\vbox{\makeheadline\pagebody\makefootline}}%
  \advancepageno%
  \ifnum\outputpenalty>-20000 \else\dosupereject\fi%
}
\catcode`@=11

\output{\facingpages}

%%%%%%%%%%%%%%%%%%%%%%%%%%%%%%%%%%%%%%%%%%%%%%%%%%%%%%%%%%%
%
% Page Frame (Headline and Footline)
%
%%%%%%%%%%%%%%%%%%%%%%%%%%%%%%%%%%%%%%%%%%%%%%%%%%%%%%%%%%%

\def\useemptyframethispage{\clearedpagetrue\displaypagefalse}
\def\usedisplayframethispage{\displaypagetrue\clearedpagefalse}

\def\useemptypageframes{
  \global\headline={\line{}}
  \global\footline={\line{}}
  \global\clearedpagefalse
  \global\displaypagefalse
}

\def\usenormalpageframes{
  \global\headline={\style{runningheader}\normalpageheadline}
  \global\footline={\style{runningheader}\normalpagefootline}
  \global\clearedpagefalse
  \global\displaypagefalse
}

\def\displaypagefootline{\style{runningheader}\hfil\folio\hfil}
\def\normalpageheadline{\ifodd\pageno\normaloddpageheadline\else\normalevenpageheadline\fi}
\def\normalpagefootline{\line{}}
\def\normaloddpageheadline{\title\hfil\folio}
\def\normalevenpageheadline{\folio\hfil\author}

%%%%%%%%%%%%%%%%%%%%%%%%%%%%%%%%%%%%%%%%%%%%%%%%%%%%%%%%%%%
%
% Pages
%
%%%%%%%%%%%%%%%%%%%%%%%%%%%%%%%%%%%%%%%%%%%%%%%%%%%%%%%%%%%

\def\nextpage{\vfil\eject\global\clearedpagetrue\global\displaypagefalse\relax}
\def\nextoddpage{\nextpage\ifodd\pageno \else \line{}\nextpage\fi}
\def\nextevenpage{\nextpage\ifodd\pageno \line{}\nextpage\else \fi}

\raggedbottom

%%%%%%%%%%%%%%%%%%%%%%%%%%%%%%%%%%%%%%%%%%%%%%%%%%%%%%%%%%%
%
% Divisions
%
%%%%%%%%%%%%%%%%%%%%%%%%%%%%%%%%%%%%%%%%%%%%%%%%%%%%%%%%%%%

\def\frontmatter{\nextoddpage\pageno=-1\useemptypageframes}
\def\mainmatter{\nextoddpage\pageno=1\usenormalpageframes}
\def\backmatter{\frontmatter}

\def\part#1{
  \nextoddpage
  \useemptyframethispage
  \baselinebox{\raggedleft\parttitle#1\par}{10}{0}
  \nextoddpage
}

\def\chapter#1{
  \nextpage
  \usedisplayframethispage
  \baselinebox{{\ifodd\pageno\raggedleft\else\raggedright\fi\chaptertitle#1\par}}{10}{1}
}

% \bookinfo is a named frontmatter or backmatter division.
\def\bookinfo#1{
  \nextpage
  \useemptypageframes
  {\centered\chaptertitle{#1}\par}
}

%%%%%%%%%%%%%%%%%%%%%%%%%%%%%%%%%%%%%%%%%%%%%%%%%%%%%%%%%%%
%
% Titles
%
%%%%%%%%%%%%%%%%%%%%%%%%%%%%%%%%%%%%%%%%%%%%%%%%%%%%%%%%%%%

% \baselinebox{content}{bottom (baselines)}{skip (baselines)}
\long\def\baselinebox#1#2#3{\vbox to #2\leading{\line{}\vfill#1}\vskip#3\leading}

\def\booktitle#1{{\style{booktitle}\titleleading\uppercase\expandafter{#1}}}
\def\authorname#1{{\style{authorname}\titleleading\uppercase\expandafter{#1}}}
\def\parttitle#1{{\style{parttitle}\titleleading#1}}
\def\chaptertitle#1{{\style{chaptertitle}\titleleading#1}}

%%%%%%%%%%%%%%%%%%%%%%%%%%%%%%%%%%%%%%%%%%%%%%%%%%%%%%%%%%%
%
% Logo
%
%%%%%%%%%%%%%%%%%%%%%%%%%%%%%%%%%%%%%%%%%%%%%%%%%%%%%%%%%%%
\newfamily{proxima}{ProximaNova}{Regular}{It}{Bold}{Regular}
\newstyle{logo}{proxima}{}{osf}{}

\newdimen\logobarthickness
\begingroup\style{logo}\global\logobarthickness=\fontdimen6\font\endgroup

\divide\logobarthickness by 16
\def\logolineskip{\vskip4\logobarthickness}

% \stackedlogo{left glue}{right glue}
\def\stackedlogo#1#2{%
  \begingroup%
  \offinterlineskip%
  \style{logo}%
  \setbox0=\hbox{DRISCOLL}%
  \vbox{%
    \hrule height \logobarthickness%
    \logolineskip%
    \copy0%
    \logolineskip%
    \hbox to \wd0{#1BROOK#2}%
    \logolineskip%
    \hbox to \wd0{#1PRESS#2}%
    \logolineskip%
    \hrule height \logobarthickness%
  }%
  \endgroup%
}

\def\leftlogo{\stackedlogo{}{\hss}}
\def\centerlogo{\stackedlogo{\hss}{\hss}}
\def\rightlogo{\stackedlogo{\hss}{}}

% \linelogo{font size}
\def\linelogo{%
  \begingroup%
  \offinterlineskip%
  \style{logo}%
  \vbox{%
    \hrule height \logobarthickness%
    \logolineskip%
    \hbox{DRISCOLL BROOK PRESS}%
    \logolineskip%
    \hrule height \logobarthickness%
  }%
  \endgroup%
}

%%%%%%%%%%%%%%%%%%%%%%%%%%%%%%%%%%%%%%%%%%%%%%%%%%%%%%%%%%%
%
% Paragraph Styles
%
%%%%%%%%%%%%%%%%%%%%%%%%%%%%%%%%%%%%%%%%%%%%%%%%%%%%%%%%%%%

\def\raggedleft{\parindent=0pt\parfillskip=0pt\leftskip=0pt plus \hsize\relax}
\def\raggedright{\parindent=0pt\parfillskip=0pt\rightskip=0pt plus \hsize\relax}
\def\centered{\raggedleft\raggedright}
\newdimen\leading \leading=15pt

\def\authorinfoheading#1{\bigskip\bigskip{\sc\centered#1\par}}
\def\authorinfoitem#1{\medskip{\centered#1\par}}
\def\connect#1#2{\authorinfoheading{#1}\authorinfoitem{#2}}
\def\genre#1{\authorinfoheading{#1}}
\def\genrebook#1{\authorinfoitem{#1}}

\def\titleleading{\linespacing{1.1}}

\def\linespacing#1{\baselineskip=#1\fontdimen6\the\font}

% TODO Create macros to set these parameters up.
\parindent=1em
\normalbaselineskip=\leading
\normalbaselines
\emergencystretch=1.5em

%%%%%%%%%%%%%%%%%%%%%%%%%%%%%%%%%%%%%%%%%%%%%%%%%%%%%%%%%%%
%
% Character Styles
%
%%%%%%%%%%%%%%%%%%%%%%%%%%%%%%%%%%%%%%%%%%%%%%%%%%%%%%%%%%%

% TODO: Make this handle nested emphasis.
%       Or give a way to go back to roman.
%       Or switch to {\rm ...} and {\it ...}.
\def\emph#1{{\it #1}}
\def\leadin#1{\nobreak\noindent{\sc #1}}

\chardef\copyright=169

%%%%%%%%%%%%%%%%%%%%%%%%%%%%%%%%%%%%%%%%%%%%%%%%%%%%%%%%%%%
%
% Microtypography
%
%%%%%%%%%%%%%%%%%%%%%%%%%%%%%%%%%%%%%%%%%%%%%%%%%%%%%%%%%%%

\begingroup\rm\pdffontexpand\font 30 30 10 autoexpand\endgroup
\pdfadjustspacing=2
\pdfprotrudechars=2

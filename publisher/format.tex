\catcode`\{=1 % begin group
\catcode`\}=2 % end group
\catcode`\$=3 % shift into and out of math mode
\catcode`\&=4 % alignment tab
\catcode`\#=6 % macro parameter prefix
\catcode`\^=7 % superscript operator
\catcode`\_=8 % subscript operator
\catcode`\^^I=10 % Treat tab as like space
\catcode`\~=13 % Tilde is active
\catcode`\^^L=13 \outer\def\^^L{\par}   % Convert form feed to \par

\countdef\nextboxregister=20 \nextboxregister=10
\countdef\nextcountregister=21 \nextcountregister=10
\countdef\nextdimenregister=22 \nextdimenregister=10
\countdef\nextskipregister=23 \nextskipregister=10
\countdef\nexttoksregister=24 \nexttoksregister=10

\outer\def\newcount#1{\global\countdef#1=\nextcountregister \advance\nextcountregister by 1}
\outer\def\newdimen#1{\global\dimendef#1=\nextdimenregister \advance\nextdimenregister by 1}
\outer\def\newskip#1{\global\skipdef#1=\nextskipregister \advance\nextskipregister by 1}
\outer\def\newtoks#1{\global\toksdef#1=\nexttoksregister \advance\nexttoksregister by 1}

\newdimen\leading

\gdef\optionalwithdefault#1#2{\if\relax#1\relax#2\else#1\fi}
\gdef\optionalwithprefix#1#2{\if\relax#2\relax\else#1#2\fi}
\gdef\fontspec{\fontbasename\optionalwithprefix{-}{\fontshape}-\fontfigures\optionalwithprefix{-}{\fontvariant}-\fontencoding\optionalwithprefix{ at }{\fontsize\relax}}
\gdef\fontshape{\optionalwithdefault{\fontweight\fontslant\fontopticalsize}{\fontdefaultshape}}

% TODO: Define these
\newcount\pageno \pageno=1

\newdimen\normalbaselineskip \normalbaselineskip=15pt plus0pt minus0pt

\newskip\bigskipsize \bigskipsize=12pt plus0pt minus0pt
\newskip\medskipsize \medskipsize=6pt plus0pt minus0pt

\newtoks\footline \footline={}
\newtoks\headline \headline={}

\def\advancepageno{\advance\pageno by 9}
\def\break{\penalty-10000 }
\def\bigskip{\vskip\bigskipsize}
\def\centerline#1{\line{\hss#1\hss}}
\def\medskip{\vskip\medskipsize}
\def\eject{\par\penalty-10000}
\def\folio{\the\pageno}
\def\dots{...}
\def\leftline#1{\line{#1\hss}}
\def\line{\hbox to\hsize}
\def\nobreak{\penalty10000 }
\def\nointerlineskip{}
\def\normalbaselines{}
\def\pagebody{\vbox to\vsize{\boxmaxdepth=\maxdepth \pagecontents}}
\def\pagecontents{\unvbox255}
\def\offinterlineskip{}
\def\raggedbottom{}
\def\rightline#1{\line{\hss#1}}
\def\thinspace{\hbox{ }}
\def\'#1{{\accent"13 #1}}

\chardef\$=`\$



% Declare a new font family. This specifies the terms to
% use to build file names for this font.
%
% The parameters are:
% 1 family tag
%     A tag to identify this font family.
% 2 file name prefix
%     The filename prefix that names this family's font
%     files. e.g. GaramondPremrPro
% 3 regular weight tag
%     The word that selects regular weight in this font's
%     file names. Usually empty, but may be Regular.
% 4 italic tag
%     The word that selects italic slant in this font's
%     file names. This is usually either It or Italic.
% 5 bold tag
%     The word that selects bold weight in this font's file
%     names. This is usually either Bd or Bold.
% 6 default shape tag
%     The word that selects the default font in this font's
%     file names. That is, it selects the font that is
%     regular weight, roman slant, and book or text optical
%     size. This is usually empty, but may be Regular.
\def\newfamily#1#2#3#4#5#6{%
  \expandafter\xdef\csname#1basename\endcsname{#2}%
  \expandafter\xdef\csname#1regularweight\endcsname{#3}%
  \expandafter\xdef\csname#1italicslant\endcsname{#4}%
  \expandafter\xdef\csname#1boldweight\endcsname{#5}%
  \expandafter\xdef\csname#1defaultshape\endcsname{#6}%
}

% Declare a new font style. This specifies the terms to use
% to build file names for this style.
%
% The parameters are:
% 1 style tag
%     A tag to identify this style.
% 2 family tag
%     Selects the font family to use for this tag.
% 3 figure style tag
%     The word that selects the desired figure style (the
%     kind of digits to use) in this font's file names.
%     This is one of:
%     - osf
%         Old style figures.
%         Variable height, proportional width.
%     - tosf
%         Tabular old style figures.
%         Variable height, fixed width.
%     - lf
%         Lining figures.
%         Fixed height, variable width.
%     - tlf
%         Tabular lining figures.
%         Fixed width, fixed height.
% 4 optical size tag
%     The word that selects the desired optical size in
%     this font's file names.
% 5 size
%     The point size at which to render this style. This
%     may be empty, in which case the font's default size
%     will be used.
\def\newstyle#1#2#3#4#5{%
  \expandafter\xdef\csname#1family\endcsname{#2}%
  \expandafter\xdef\csname#1opticalsize\endcsname{#3}%
  \expandafter\xdef\csname#1figures\endcsname{#4}%
  \expandafter\xdef\csname#1size\endcsname{#5}%
}

\def\style#1{%
  \edef\fontfamily{\csname#1family\endcsname}%
  \edef\fontbasename{\csname\fontfamily basename\endcsname}%
  \edef\fontdefaultshape{\csname\fontfamily defaultshape\endcsname}%
  \edef\fontopticalsize{\csname#1opticalsize\endcsname}%
  \edef\fontfigures{\csname#1figures\endcsname}%
  \edef\fontsize{\csname#1size\endcsname}%
  \rm%
}

\gdef\withfontbasename#1{\edef\fontbasename{#1}}
\gdef\withfontencoding#1{\edef\fontencoding{#1}}
\gdef\withfontfigures#1{\edef\fontfigures{#1}}
\gdef\withfontopticalsize#1{\edef\fontopticalsize{#1}}
\gdef\withfontshape#1#2#3{\withfontweight{#1}\withfontslant{#2}\withfontopticalsize{#3}}
\gdef\withfontslant#1{\edef\fontslant{#1}}
\gdef\withfontvariant#1{\edef\fontvariant{#1}}
\gdef\withfontweight#1{\edef\fontweight{#1}}

\gdef\selectfont{\font\currentfont=\fontspec\currentfont\definefontexpandability}

\gdef\rm{\withfontweight{\csname\fontfamily regularweight\endcsname}\withfontslant{}\withfontvariant{}\withfontencoding{t1}\selectfont}
\gdef\it{\withfontslant{\csname\fontfamily italicslant\endcsname}\selectfont}
\gdef\bf{\withfontweight{\csname\fontfamily boldweight\endcsname}\selectfont}
\gdef\sc{\withfontvariant{sc}\selectfont}
\gdef\tl{\withfontvariant{titling}\withfontfigures{tlf}\selectfont}
\gdef\sym{\withfontencoding{ts1}\selectfont}

%%%%%%%%%%%%%%%%%%%%%%%%%%%%%%%%%%%%%%%%%%%%%%%%%%%%%%%%%%%
%
% Text and Title Styles
%
%%%%%%%%%%%%%%%%%%%%%%%%%%%%%%%%%%%%%%%%%%%%%%%%%%%%%%%%%%%

\def\aheadsize{36pt} % Book and part titles
\def\bheadsize{24pt} % Author name
\def\cheadsize{18pt} % Chapter titles and publisher logo
\def\bodytextsize{11pt}
\def\smalltextsize{10pt}
\def\runningheadersize{8pt}

\def\definestyles{%
  \newstyle{booktitle}{\booktitlefamily}{\aheadopticalsize}{tlf}{\aheadsize}%
  \newstyle{authorname}{\authornamefamily}{\bheadopticalsize}{tlf}{\bheadsize}%
  \newstyle{parttitle}{\parttitlefamily}{\aheadopticalsize}{osf}{\aheadsize}%
  \newstyle{chaptertitle}{\chaptertitlefamily}{\cheadopticalsize}{osf}{\cheadsize}%
  \newstyle{bodytext}{\bodytextfamily}{\bodytextopticalsize}{osf}{\bodytextsize}%
  \newstyle{smalltext}{\smalltextfamily}{\smalltextopticalsize}{osf}{\smalltextsize}%
  \newstyle{runningheader}{\runningheaderfamily}{\runningheaderopticalsize}{osf}{\runningheadersize}%
}

\newdimen\leading

% TODO: Move this elsewhere
\leading=15pt

%%%%%%%%%%%%%%%%%%%%%%%%%%%%%%%%%%%%%%%%%%%%%%%%%%%%%%%%%%%
%
% Textblock Size and Position on the Page
%
%%%%%%%%%%%%%%%%%%%%%%%%%%%%%%%%%%%%%%%%%%%%%%%%%%%%%%%%%%%

\pdfhorigin=0pt
\pdfvorigin=0pt

\newdimen\edgemargin
\newdimen\spinemargin
\def\setspinemargin#1{
  \edgemargin=#1
  \spinemargin=\pdfpagewidth
    \advance\spinemargin by -\hsize
    \advance\spinemargin by -#1
}

% Use edge margin for left/even pages,
% spine margin for odd/right pages
\def\leftmargin{\ifodd\pageno\edgemargin\else\spinemargin\fi}

% TODO: Move these assignments elsewhere
\pdfpagewidth=5 true in   % page width
\pdfpageheight=8 true in  % page height
\hsize=22pc               % textblock width
\vsize=39pc               % textblock height
\voffset=4.5pc            % top margin
\setspinemargin{4.5pc}    % spine margin

%%%%%%%%%%%%%%%%%%%%%%%%%%%%%%%%%%%%%%%%%%%%%%%%%%%%%%%%%%%
%
% Output Routine
%
%%%%%%%%%%%%%%%%%%%%%%%%%%%%%%%%%%%%%%%%%%%%%%%%%%%%%%%%%%%

\def\runningheadline{\line{}}
\def\runningfootline{\line{}}
\def\useclearedframethispage{\global\headline={\clearedheadline}\global\footline={\clearedfootline}}
\def\usedisplayframethispage{\global\headline={\displayheadline}\global\footline={\displayfootline}}
\def\userunningpageframe{\global\headline={\runningheadline}\global\footline={\runningfootline}}

% Write one page (headline, textblock, and footline).
\def\facingpages{%
  \shipout\vbox{\moveright\leftmargin\vbox{\makeheadline\pagebody\makefootline}}%
  \userunningpageframe%
  \advancepageno%
  \ifnum\outputpenalty>-20000 \else\dosupereject\fi%
}

\output{\facingpages}

\newdimen\headlineskip

% Position the headline 2 baselines above the textblock's first baseline.
% TODO: ?? Make this a macro so that it can be called after
% \runningheadersize and \leading are defined?
\headlineskip=\topskip % skip to textblock's first baseline
\advance\headlineskip by -2\leading % skip two lines above first baseline
\advance\headlineskip by -\runningheadersize % skip up to make room for headline

\def\makeheadline{\vbox to 0pt{\vskip\headlineskip\line{\the\headline}\vss\nointerlineskip}}
% Write the footline 2 baselines below the textblock's last baseline.
\def\makefootline{\baselineskip=2\leading\lineskiplimit=0pt\line{\the\footline}}

%%%%%%%%%%%%%%%%%%%%%%%%%%%%%%%%%%%%%%%%%%%%%%%%%%%%%%%%%%%
%
% Page Frame (Headline and Footline)
%
%%%%%%%%%%%%%%%%%%%%%%%%%%%%%%%%%%%%%%%%%%%%%%%%%%%%%%%%%%%

% Running headline and footline for cleared pages
% (pages begun by \nextpage, \nextoddpage, and \nextevenpage).
% You may want to change these for debugging.
\def\clearedheadline{\line{}}
\def\clearedfootline{\line{}}

% Running headline and footline for display pages
% (pages that begin with a major display item such as a part or chapter title)
\def\displayheadline{\line{}}
\def\displayfootline{\style{runningheader}\sc\hfil\folio\hfil}

% Running headline and footline after user calls \useemptypageframe.
% By default, this format uses empty frames for frontmatter and backmatter pages.
% You may want to change these for debugging.
\def\emptyheadline{\line{}}
\def\emptyfootline{\line{}}

% Running headline and footline for most content pages.
% We use different headlines on even and odd pages.
% Redefine to suit your formatting needs.
\def\normalheadline{\style{runningheader}\sc\ifodd\pageno\normaloddpageheadline\else\normalevenpageheadline\fi}
% No footline on normal pages
\def\normalfootline{\line{}}
% Odd (right) pages: title at inner margin, page number at outer margin
\def\normaloddpageheadline{\style{runningheader}\sc\title\hfil\folio}
% Even (left) pages: page number at outer margin, author name at inner margin
\def\normalevenpageheadline{\style{runningheader}\sc\folio\hfil\author}

\def\useemptypageframe{
  \gdef\runningheadline{\emptyheadline}
  \gdef\runningfootline{\emptyfootline}
  \userunningpageframe
}

\def\usenormalpageframe{
  \gdef\runningheadline{\normalheadline}
  \gdef\runningfootline{\normalfootline}
  \userunningpageframe
}

%%%%%%%%%%%%%%%%%%%%%%%%%%%%%%%%%%%%%%%%%%%%%%%%%%%%%%%%%%%
%
% Pages
%
%%%%%%%%%%%%%%%%%%%%%%%%%%%%%%%%%%%%%%%%%%%%%%%%%%%%%%%%%%%

\def\nextpage{\vfil\eject\useclearedframethispage}
\def\nextoddpage{\nextpage\onoddpage}
\def\nextevenpage{\nextpage\ifodd\pageno \line{}\nextpage\else \fi}
\def\onoddpage{\ifodd\pageno \else \line{}\nextpage\fi}
\raggedbottom

%%%%%%%%%%%%%%%%%%%%%%%%%%%%%%%%%%%%%%%%%%%%%%%%%%%%%%%%%%%
%
% Divisions
%
%%%%%%%%%%%%%%%%%%%%%%%%%%%%%%%%%%%%%%%%%%%%%%%%%%%%%%%%%%%

\def\frontmatter{\nextoddpage\pageno=-1\useemptypageframe}
\def\mainmatter{\nextoddpage\pageno=1\usenormalpageframe}
\def\backmatter{\frontmatter}
\def\endbook{\nextoddpage\end}

\def\part#1{
  \nextoddpage
  \baselinebox{\raggedspine\parttitle{#1}\par}{10}{0}
  \nextoddpage
}

\def\chapter#1{
  \nextpage
  \usedisplayframethispage
  \baselinebox{\raggedspine\chaptertitle{#1}}{10}{1}
}

% \bookinfo is a named frontmatter or backmatter division.
\def\bookinfo#1{
  \nextpage
  \useemptypageframe
  {\centered\chaptertitle{#1}\par}
}

%%%%%%%%%%%%%%%%%%%%%%%%%%%%%%%%%%%%%%%%%%%%%%%%%%%%%%%%%%%
%
% Titles
%
%%%%%%%%%%%%%%%%%%%%%%%%%%%%%%%%%%%%%%%%%%%%%%%%%%%%%%%%%%%

% \baselinebox{content}{bottom (baselines)}{skip (baselines)}
\long\def\baselinebox#1#2#3{\vbox to #2\leading{\line{}\vfill#1}\vskip#3\leading}

\def\booktitle#1{{\style{booktitle}\titleleading\uppercase\expandafter{#1}\par}}
\def\authorname#1{{\style{authorname}\titleleading\uppercase\expandafter{#1}\par}}
\def\parttitle#1{{\style{parttitle}\sc\titleleading#1\par}}
\def\chaptertitle#1{{\style{chaptertitle}\sc\titleleading#1\par}}

%%%%%%%%%%%%%%%%%%%%%%%%%%%%%%%%%%%%%%%%%%%%%%%%%%%%%%%%%%%
%
% Logo
%
%%%%%%%%%%%%%%%%%%%%%%%%%%%%%%%%%%%%%%%%%%%%%%%%%%%%%%%%%%%
\def\logotextsize{\cheadsize}
\newfamily{proxima}{ProximaNova}{Regular}{It}{Bold}{Regular}
\newstyle{logo}{proxima}{}{osf}{\logotextsize}

\newdimen\logobarthickness \logobarthickness=\logotextsize
\divide\logobarthickness by 16
\def\logolineskip{\vskip4\logobarthickness}

\def\logostack#1#2{%
  \setbox0=\hbox{DRISCOLL}%
  \copy0%
  \logolineskip%
  \hbox to \wd0{#1BROOK#2}%
  \logolineskip%
  \hbox to \wd0{#1PRESS#2}%
}

\def\logo#1{
  \begingroup%
  \offinterlineskip%
  \style{logo}%
  \vbox{%
    \hrule height \logobarthickness%
    \logolineskip%
    #1%
    \logolineskip%
    \hrule height \logobarthickness%
    }%
  \endgroup%
}

\def\leftlogo{\logo{\logostack{}{\hss}}}
\def\centerlogo{\logo{\logostack{\hss}{\hss}}}
\def\rightlogo{\logo{\logostack{\hss}{}}}
\def\linelogo{\logo{\hbox{DRISCOLL BROOK PRESS}}}

%%%%%%%%%%%%%%%%%%%%%%%%%%%%%%%%%%%%%%%%%%%%%%%%%%%%%%%%%%%
%
% Paragraph Styles
%
%%%%%%%%%%%%%%%%%%%%%%%%%%%%%%%%%%%%%%%%%%%%%%%%%%%%%%%%%%%

\def\raggedleft{\parindent=0pt\parfillskip=0pt\leftskip=0pt plus \hsize\relax}
\def\raggedright{\parindent=0pt\parfillskip=0pt\rightskip=0pt plus \hsize\relax}
\def\centered{\raggedleft\raggedright}
\def\raggedspine{\ifodd\pageno\raggedleft\else\raggedright\fi}

\def\authorinfoheading#1{\bigskip\bigskip{\sc\centered#1\par}}
\def\authorinfoitem#1{\medskip{\centered#1\par}}
\def\connect#1#2{\authorinfoheading{#1}\authorinfoitem{#2}}
\def\genre#1{\authorinfoheading{#1}}
\def\genrebook#1{\authorinfoitem{\emph{#1}}}

\def\titleleading{\linespacing1}

\def\linespacing#1{\baselineskip=#1\fontdimen6\the\font}

% TODO Create macros to set these parameters up.
\parindent=1em
\normalbaselineskip=\leading
\normalbaselines
\emergencystretch=1.5em

%%%%%%%%%%%%%%%%%%%%%%%%%%%%%%%%%%%%%%%%%%%%%%%%%%%%%%%%%%%
%
% Character Styles
%
%%%%%%%%%%%%%%%%%%%%%%%%%%%%%%%%%%%%%%%%%%%%%%%%%%%%%%%%%%%

% TODO: Make this handle nested emphasis.
%       Or give a way to go back to roman.
%       Or switch to {\rm ...} and {\it ...}.
\def\emph#1{{\it #1\/}}
\def\leadin#1{\nobreak\noindent{\sc #1}}

\def\copyright{{\sym\char"A9}}

%%%%%%%%%%%%%%%%%%%%%%%%%%%%%%%%%%%%%%%%%%%%%%%%%%%%%%%%%%%
%
% Microtypography
%
%%%%%%%%%%%%%%%%%%%%%%%%%%%%%%%%%%%%%%%%%%%%%%%%%%%%%%%%%%%

\gdef\definefontexpandability{\pdffontexpand\font 30 30 10 autoexpand}
\pdfadjustspacing=2
\pdfprotrudechars=2

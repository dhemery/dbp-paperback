\def\previewtitle{Tailor’s Tears}
\def\previewauthor{Dale Hartley Emery}

\previewannouncementpage

\previewpage

\baselinebox{\typesetchapterheading{\previewtitle}}{10}{}

\baselinebox{\typesetsceneheading{A Tattered Wedding Dress}}{2}{1}
\useemptypageframe

\leadin{There are three things} a tailor needs above all others: cloth, needles,
and a fresh supply of tears. Clarke Whiteley was all out of tears. He
looked at Dulcie Byers’s wedding dress, in tatters on the cutting table,
and wondered whether he could retire on the money in his savings
account. He decided he could not.

Clarke’s workshop was empty now, his boutique closed. The cheap
battery-powered wall clock ticked heavily, and the mid-afternoon
humidity hung in the room like a gas leak.

Dulcie Byers was likely halfway home by now, screaming into her cell
phone at her unworthy fiancé that the tailor he recommended, the tailor
he \emph{insisted} that she use, had ruined her \$3,000 wedding dress.
Maybe she would scream, as she had at Clarke, that he had ruined her
life.

She had a point.

Hue Hawthorne, the unworthy fiancé, would call Clarke to find out what
had happened, to find out how to make this right. That conversation
would lead to nowhere, and he too would yell at Clarke, about trust,
about putting my faith in you, about third and fourth chances, about I
knew you would let me down when it really mattered.

Ten years earlier Clarke had shepherded Hue through his senior year at
Brown University, tutoring, writing papers, stealing advance copies of
upcoming tests, ensuring Professor Combover that surely he was mistaken,
surely it was not Hue Hawthorne he had spotted with tender young Fiona
Combover leaving Rosa’s Ristorante, because on Thursday evening Hue had
been in the library with Clarke, studying diligently for the poli-sci
midterm.

Hue was nothing if not grateful, but gratitude had a shelf life. The
cheesy clock on Clarke’s wall ticked past the expiration date.

The boutique’s phone rang, and Clarke answered before the first ring
died away. “Hey, Hue.”

“You son of a cur. What have you done to my daughter-in-law?” It was not
Hue. It was the elder Hawthorne. Assemblyman Gorance Goldsmith Hawthorne
III. Old Gory had never liked Clarke.

“I’m sure she’s told you the whole story herself,” Clarke said, sure she
had done no such thing.

“The poor girl was unconsolable. Incoherent. Couldn’t say three words
without breaking into great gasping sobs. How could you let her drive in
that state?”

That had been a mistake. It would also have been a mistake to try to
reason with an enraged bride-to-be in a room full of scissors.

“I do hope she gets home safely.”

Gory shouted, “You do not want to play games with me, you little weasel.
I advise you to repair that dress with all possible haste.”

“That won’t be possible—”

“Why not? What sort of incompetent tailor are you?”

“I am competent. I am not magical.” Which was exactly the opposite of
the truth. And that was what created this whole mess. He should not have
imbued Hue’s fiancée’s wedding dress with magic. Not with a maiden aura.
Not four days before the wedding.

And certainly not without conferring with the bride.

“Is it money? Is that it? Four days before the wedding and you stoop to
extortion?”

“No, sir, nothing like that,” Clarke said. “I can’t repair it for any
amount of money. It’s been torn to shreds.” And imbued to glow with the
subtle pink aura of virginity at her moment of happiness.

Assuming she was actually a virgin.

Gory screamed, “Why in the name of all the gods would you do such a
thing?”

“Actually, Mister Hawthorne, she was the one who—”

“‘He ruined it.’ Those were the three words she was able to choke out
between sobs. ‘He ruined it.’”

Clarke had to admit that she was right. Sure, she was the one who had
torn the dress to shreds, but by then it was already ruined.

“I assure you it was an accident.” Not so much an accident as an
oversight. He should have known better than to take the prospective
mother-in-law’s word for something as potentially catastrophic as this.
Even if she was the wife of the most powerful politician in the county.
Especially then.

“Well, for god’s sake, man, make it right.” Gory’s voice no longer
sounded angry. More like pleading.

Make it right. It sounded like such a reasonable plea. “I will do what I
can, Mister Hawthorne,” Clarke said. “I promise—”

“You would be surprised how little I value your promises right now,”
Hawthorne said, and Clarke could just picture the man spitting the words
through clenched teeth. “But do make this right. I am feeling an
exaggerated sense of stress in my life right now, and I would not want
that to affect my professional judgment in the upcoming zoning board
meetings. A happy, boring wedding would be a great relief. Am I making
myself clear?”

Clarke’s Corner Clothier was on the western edge of the commercial zone,
next to a large undeveloped lot. Hawthorne was golf buddies with Mathias
McCoy, the owner of the lot, who wanted to expand his holdings eastward
and build an apartment complex or a strip mall or a putt putt golf
course. The five-year zoning plan was up for review next month. The
chairman of the assembly’s zoning committee was none other than Gorance
Goldsmith Hawthorne III.

Yes, Gory was making himself clear. Clarke bristled at the threat. Gory
could have appealed to Clarke’s human decency. Or to his sense of
business propriety. Or even to guilt. The implication that Clarke would
do the right thing only under threat was an insult.

But not entirely undeserved.

Clarke said, “I think we’re all hoping for a happy wedding, sir.”

Gory was right. Clarke had to make this right.

He had no idea how he was going to do that. But it had to start with
Dulcie.

\previewavailablepage

\chapter{Cycle 1}

\scene{Mamie and Pickles}

\leadin{Pickles belted out a yowl} that, to Mamie Levine’s ears, put the alley cats and their nightly startle-fest to shame.

Outside her open living room window and three stories below, the alley cats proved her wrong by screaming so nearly in unison that their voices beat together in alternating peaks of cancellation and summation.

Pickles spun three or four complete rotations, at the same time flopping through three or four barrel rolls, howling madly through the whole acrobatic impossibility.

“Good lord, Pickles,” Mamie said. “You’re acting as if you were caught in a blender.”

Pickles spun and rolled a few more times, jumped straight up, landed on all fours and zipped like a cartoon across the living room carpet into the bedroom.

The cats outside were joined now by dogs or wolves or coyotes.

What in blue blazes had gotten into those animals? Was there an earthquake coming? Here in Chicago?

Mamie looked into the bedroom, where Pickles was tearing at the bedspread with her claws. Bits of fluff flitted into the air.

That was going to take a lot of work to patch.

Pickles darted off the bed toward the window. Mamie heard the harsh scrabbling of Pickles’s usually dainty claws. Then something crashed (there goes the curtain rod), and Pickles burst, howling, down the hallway toward the kitchen. The howling cut off suddenly, followed by a loud bump (cat into cabinet) and a squeeky “bip” from Pickles.

“Oh, dear,” Mamie said. “I just waxed that floor yesterday.”

Mamie looked again into the bedroom. Pickles’s furry fury hadn’t disturbed the rest of the bedspread, which was still tucked primly under and folded primly over the pillows.

Mamie cocked her head and looked at the neatly made bed. Wasn’t she just in there, snuggled into bed, reading J. D. Robb’s latest futuristic detective romance novel? When had she made the bed?

And she couldn’t remember coming into the living room. One minute she had been in bed reading, and the next she had been in the living room watching Pickles go bonkers.

Another fugue. Oh, dear.

The noise outside was louder now. Mamie could not hear Pickles.

“Pickles?”

She walked down the hall to the kitchen. Pickles sat on his haunches, facing toward Mamie, panting.

Her cat was panting?

Mamie noticed that the floor was wet. In the far corner the mop stood in the mop bucket. She could smell the fresh wax.

Fugue mopping? That couldn’t be a good sign.

Outside the cats screamed and the dogs howled. And there was more. People screaming. Horns honking.

Mamie heard a siren in the distance.

Then the air raid warning siren two blocks away burst into song.

What in the blue blazes? Mamie thought.



\scene{Olin in Traffic}

\leadin{Something (a bird?)} loomed very close above Olin Montgomery’s head. He ducked forward instinctively and smashed his forehead solidly on the empty air in front of him, the shock sending a shower of sparks over his field of vision.

What had he hit his head on, his golf club? No, he could feel the club in his hands, and his head had hit something in between.

Between my hands?

Olin opened his eyes and screamed. He was hurtling down the highway (I’m on the 10) at full speed. Cars in front of him were moving all the wrong ways, traveling sideways, slowing, swerving, crashing into each other in a mad jumble. A hundred car horns bleated.

Olin was bearing down too fast on an old white pickup truck that was spinning in the road in front of him, spraying lawn care equipment in all directions. He stomped on the brake and yanked the steering wheel (not my five iron) sharply to the left. The rear of his Mercedes SUV snaked around to the right, jerked on something (pothole), swerved back behind and around to the left. Olin was thrown first to the right, sharply against the suddenly slackless seatbelt, then to the left against the door. A sheet of white (side air curtain) flashed in the corner of his eye, and his head hit something and twisted. The SUV tilted up on its left wheels. For a second it swivelled down the highway, wrenching the steering wheel from Olin’s hands, then tipped onto its side. The door glass splintered and the fragments spun and twisted as the car scraped and screeched roof first along the ridged concrete.

Olin could feel the SUV’s momentum slowing. I’m going to get out of this okay.

A sickening crunch hurled Olin toward the roof of the car, his seat belt slicing into his legs. The roof buckled and struck his head, bending him double as he dangled from the car seat.

Olin’s vision blended from black to red to brown, the brown of the car’s interior decor. His eyes hurt, and the top of his head felt like it was split open.

His hands still gripped the steering wheel.

Olin released the steering wheel and brought a hand to the top of his head. It didn’t feel wet. He looked at his hand. No blood. But Christ his head hurt.

Around him, beyond the crumpled confines of his SUV, the sounds of screeching tires and crumpling metal diminished. People were screaming, and perhaps had been screaming for some time. Some of the screams sounded like screams of pain. Others sounded like something worse.

“I need help over here!” someone shouted.

What the hell had loomed over him? It wasn’t a bird. It was… It was the roof of his car. As he swung his five iron back for an approach shot on the eighteenth hole at Hillcrest, the roof of his car had suddenly loomed above his head, and he suddenly he had been shooting down the 10.

Concussion. Had he passed out? He may have, when the roof had cracked him on the head.

The car’s digital clock, angled oddly above him, read 2:31 pm.



\scene{Erika in Labor}

\leadin{Erika Howard woke up screaming.} Screaming and in labor.

What the hell? She had given birth just a few hours ago.

Doctor Morris stood between her legs. He looked concerned.

How had they gotten her back on the table without waking her?

“What just happened?” Morris said.

The nurse, the plain one, screamed. What was her name? Erika couldn’t remember. The other nurse, the pretty one, fainted. Erika knew his name. He was Tom.

Outside the room several people screamed. Other moms in labor? No, one of them sounded like a man.

“What is it?” Erika said. “Twins?”

She hadn’t expected twins. She wasn’t expecting twins. The ultrasounds had showed one child. How could they have missed a second baby?

“I don’t know,” said Morris. “I’m not feeling well.”

“Well, I’m not stopping now, Doc!,” said Erika. “This train has left the station.”

Janet looked down at Tom, then at the doctor with a puzzled look on her face. “What’s happening?”

“I don’t…” Morris stepped back from between Erika’s legs. He picked up the chart that hung on a hook at the end of the birthing table.

“Howard,” he said. He looked at Erika, then again at the chart. “Didn’t we already…” Morris swayed, then put a hands on the table to steady himself.

“Yes, we fucking already,” Erika said. Her son Kyle had been born hours earlier, at 11 am.

She cramped again. The pain was excruciating. She screamed.

Morris repositioned himself. “Four centimeters?” he said. It was a question.

Jesus, Morris seemed to have no idea what he was doing.

“How the fuck did you not know about the second one!”

Morris didn’t answer. Instead he looked at Janet. “Did something just happen?”

“I was at a movie with Bob,” Janet said. “And then I was here.”

Morris looked at his watch and blinked. He held his wrist to his ear as if listening for the watch to tick. He stared again at the watch. “Digital,” he said, then brought it again to his ear.

He looked at Janet. “What time is it?”

Janet looked at her watch, then held it to her ear. She looked up at the wall. Erika followed her gaze.

The clock read 3:31.

“I don’t understand,” Janet said.

Morris said, “What day?”

“What day?” Erika said. “How fucking long was I asleep?”

“Sunday?” Janet said.

“No, I’m off on Sunday,” Morris said. “My watch says Saturday. 3:33 pm.”

“Hey, yo, who gives a crap about your watch? I’m giving birth here.”

Janet looked at her watch. “Mine, too.”

Morris looked up at the clock on the wall. “I was watching the Diamondbacks.”

“Are you going to deliver this goddamned baby or not?”

“I thought I did.”

What the hell?

“My second baby, you fucking quack.”

Morris pointed between Erika’s legs and frowned. “You’ve never had a baby before.”

“Don’t you tell me I’ve never had a baby,” Erika yelled. “He’s in your nursery.”

“I’m really not feeling well,” Morris said, and turned and walked out of the room.

Janet stared after him.

Someone screamed from down the hall.

Erika said, “Will somebody please tell me what is going on?”

“Oh, I don’t think so,” Janet said quietly. “I don’t think that’s going to happen.”



\scene{Jude Leaps}

\leadin{Jude Elliott was surprised} to be here, straddling the railing of the twelfth-floor fire escape more than a hundred feet above the alley behind his apartment building.

Jude Elliott was surprised to be alive.

He had leaped from the fire escape. In the two and a half seconds that followed, he just had time to think, Oh, good, I’m going to miss the dumpster.

He didn’t remember hitting the ground. And yet he was pretty sure that he had hit the ground. How could he not?

Jude lifted his left leg over the railing. Holding on with both hands, he leaned forward and looked down to the empty alley.

He released his grip on the railing. Gravity pulled him forward and downward. In the two and a half seconds that followed, he just had time to think, Headline: Clumsy teen throws self at ground and misses.

\chapter{Cycle 2}

\scene{Mamie and Pickles}

\leadin{Mamie Levine found herself} in the living room again. Now, what the dickens?

Pickles started to meow, and choked it off in a startled “meep.”

“I’ve lost my marbles, Pickles, dear. Have you seen my marbles?”

Pickles sat on his haunches, looking up at Mamie. His tongue lolled out of his mouth. And he was panting again.

“Oh, dear, you’re all upset again.” Mamie bent down and picked up the cat, who went limp in her arms. “You’re shaking. What’s gotten into you?”

Mamie rolled Pickles onto his back, but he wriggled himself upright again. He usually loved having his belly rubbed. But he was still shaking. Shaking and panting.

“You’re really upset, aren’t you Pickles?” Mamie said.

Pickles just panted.

“Maybe you’d like some milk. That always helps.”

Mamie carried Pickles into the kitchen, stroking him and humming softly. As she stepped onto the kitchen floor, her foot slipped and her legs splayed, tossing cat one way and Mamie the other.

“What the ding dang?” Maybe squeaked.

The floor was wet. The whole floor was wet, including a few puddles in the low spots of the uneven linoleum.

And in the corner, the mop stood in a blue bucket.

Mamie stared at the mop, trying to make sense of what she was seeing. Finally, she said, “Well, Pickles, I guess if I’m going to go crazy, I might as well be have a clean house.”

Pickles looked at Mamie and quiverred.



\scene{Olin in Traffic}

\leadin{Olin noticed the sound first.} He had been sitting quietly in his living room. At least, outside of his head had been quiet. Inside his head was a jumble of words. Not quite coherent thoughts, but lots of words, as he tried to cope with the paralyzing fear he had felt for the past 29 hours. He had thought several times to grab his clubs out of his closet—in the time loop, his clubs had reappeared in his closet, magically transported from the 18th hole at Hillcrest—and drive out to the golf course. But whenever he imagined himself winding back for a tee shot, an image that usually calmed him profoundly, he began to shake uncontrollably.

For the past 29 hours he had alternated between that unfairly fear inducing image and an image of a life in which he could never play golf again, for fear of finding himself suddenly hurtling down the 10 at 70 miles per hour.

He had moments of respite, moments of relative relief from the images in his head.

And it was in one of those moments of respite, one of those moments of relative calm, that Olin Montgomery found himself again in his Mercedes, hurtling southward on the 10 at 70 miles per hour.

The white pickup truck in front of him, swerved and went up on two wheels. A bucket of rakes tumbled ofer the side, bounced on the asphalt, and flipped in mid-air. As the bucket flew past Olin’s SUV, the rakes were all miraculously still in it.

The pickup then smacked back down onto all four wheels. It continued to swerve, but did not, as it had last time, go into an uncontrolled spin.

One of the white pickup truck’s taillights lit up. Olin jammed his foot on the brake pedal. He wanted to look in the rear view mirror to see whether anyone was bearing down on him, but he didn’t dare take his eyes off of the chaos in front of him.

Everywhere in front of the white truck and beside it taillights were lighting up. There were, Olin noticed with a kind of shock, no collisions. Or, at least, none that he could see.

Traffic in front of him slowed, rapidly but orderly. Olin peeked in his mirror. Traffic behind was slowing, too.

He was going to get out of this one unscathed.

And, he realized to his horror, there was probably going to be a next time, and a next.

But he had managed this time, as had other people. He was learning, and they were learning.



\scene{Erika in Labor}

\leadin{Erika Howard woke up screaming,} again, and in labor, again.

Tom, the nurse, said, “Uh oh…”

Erika said, “I thought you weren’t supposed to say `uh oh’ in an operating room.”

“It looks as if we’re going to have to do this all over again,” said Doctor Obvious.

“Any chance we can take the fast lane this time, doc?”

“The fast lane?”

“Grab your butcher knife and go in after him,” Erika said. “I mean, I love the kid and all, but I don’t think I want to go through labor three times for him.”

“I have a hunch,” said Doctor Morris, “that this won’t be the last time.”

“Oh, don’t you tell me that, you son of a bitch.” Erika propped herself up on her elbows and shook her head decisively from side to side. “Don’t you fucking tell me that.”



\scene{Jude Jumps}

\leadin{Jude Elliott thought,} Man, death is weird. He had heard that at the moment of death your life passes before your eyes. But he didn’t know it meant that you have to live the moment of death itself over and over.

He looked down from the fire escape. Did he have to jump? What if he didn’t jump? Would he still be dead? He didn’t know the rules. Death was weird. He hadn’t expected this, and he wasn’t sure he wanted to keep doing it, even if he had to.

Maybe this was some kind of cosmic second chance. A test, to see whether you really, really, really wanted to be dead. If you jumped enough times, you got to stay dead.

Jude swung his leg over the railing.

“Hey!” a voice called from below.

Jude looked down and didn’t see the caller.

“Over here,” the voice said. Toward the end of the alley was a man coming out of an open door in the building across the alley. In his hand was a black garbage back with a red tie.

Power tie, Elliott thought, and laughed.

“For Christ’s sake, stop jumping,” the man said. “For my sake, even. Jesus you make a godawful mess. I threw up both times. So stop it, okay?”

Jude said, “It’s a test.”

“What?”

Louder, Jude said, “It’s a test. To see if I really want to go through with it.”

“I don’t think it’s a test, man. The news said it was some kind of worldwide time thing, some kind of hiccup in time. I don’t think this is about you.”

“Just in case,” Jude said, and stepped forward into empty air.



\scene{Rynn in Space}

\leadin{Rynn Haney floated} in space again.

Houston had warned her that it might happen again. Of course, they had no explanation for what was happening, but NASA engineers were experts in risk management, and when their people were in space they left no possibility unconsidered. They weighed the possibility—guessed, in other words, Rynn thought—that the time loop would happen again. Very low probability, given that CERN had shut the Large Hadron Collider down, but not zero probability. They weighed the possibility that if it were to happen again it would happen at the same time. This was harder to gauge, given that nobody knew what had happened at the later end of the loop to cause time to revert to the earlier end, so they could not predict when, if the triggering event happened at all, it would happen again.

Rynn had been on edge for the duration of the last cycle, unable to sleep or even rest. She was trained in a zillion kinds of specific emergencies, and trained in coping emotionally and intellectually with the unpredictable emergencies—what former Sec Def Rumsfeld had called the unknown unknowns.

Always be ready for anything. That was the easy way to say it. But that was beyond Rynn. It was beyond any human, she thought. Nobody could stand to be on edge all the time, waiting for anything to happen. So you made assumptions and you acted as if they were true. And when the assumption went spinning out the window—or out of the space shuttle, as it were—you called on your training. Observe, orient, decide, act. That was the mantra, that was Colonel Boyd’s OODA loop. Observe what is happening around you, orient yourself, make a decision, and act. Easy to say, not so easy to do. Each step was filled with traps. In an emergency, there is too much information to observe, so you focus (if you can) on the most salient features. This wasn’t always easy, or even possible, because you may not know what features of the situation are salient. Orienting depended on being able to pick out the salient features of the situation; but picking out the salient features depended on being oriented. A catch twenty two. Deciding was obviously problematic, especially on your first cycle through the OODA loop. How can you make a good decision with horribly inadequate information and an unstable orientation? But that was why you had to decide quickly, so that you could act. Acting would produce new information that you could observe on your second pass through the OODA loop.

Observe, orient, decide, act.

It was very helpful if you could orient yourself before jumping into the emergency, if you had time to orient yourself.

Bob Lyman, mission ground director in Houston, had prepared her with a countdown. They didn’t know, of course, whether the time anomaly would happen again. Nobody knew. But to be on the safe side, they had made the assumption—knowing that it was an assumption—that if the loop were going to happen, it would happen at the same time.

“Atlantis, anomaly confirmed,” the voice said in her ear.

“Bonus EVA confirmed,” Rynn said, “How many E tickets do I have left?”

Bob said, “Prepare for reboarding.”

“I don’t think that’s a good idea, Bob,” Rynn said.

“We need you safe, Rynn. Prepare to reboard.”

“What about this gash? I have to spackle it.”

“We can decide that once you’re inside. Prepare to reboard. Please, Rynn. We can decide once you’re inside.”

“I’ve opened the NOAX already. It’s use it or lose it time. We don’t get a second chance to repair this gash.”

“We need you safe Rynn. Prepare to reboard. That’s an order.”

Time for step three, Rynn thought. Decide.

“Sir, with all due respect, if I don’t fix the skin now, in the next 20 minutes, I’m putting all seven of us at risk.”

Several seconds passed in silence.

Rynn wanted to make sure Houston was crystal clear about her intentions. “I’m going to patch the skin, Bob.”



\scene{Alpoo’s Blog}

\leadin{Stay tuned} for my serial novel. Posted August 9, 2008, 11:00 am by AlPoo.

Hey, kiddies, guess what? I’m going to write my novel after all.

For those of you who missed my rant yesterday, let me summarize. Waaahhhh! I can’t write my novel because the time loop erases everything I write. Waaahhhh!

I think that gives the gist of it. As for the details, well, those are gone like the pages of my novel. C’est la guerre, or however you spell that. Them’s the breaks.

But I figured it out: I can still write my novel. But I’ll publish it a scene at a time, right here on AlPoo Saves the Day. Every day I’ll write another scene and publish it here. If you want to read it, you’ll just have to keep up.

Dostoevsky wrote serials, and he did okay. Dickens did okay, too, with serials, at least if Miss Breckenhall, my ninth grade English teacher, told me the truth. And even Stephen King wrote a serial once.

So I’m in good company. And anyway with the time loop there’s no other way to do it.

Once again blogs are the future of publishing. Once again the brilliant and handsome AlPoo saves the day.

Oh, and Miss Breckenhall… If you’re out there somewhere reading this, you should know that I still pine for you just like—excuse me—just as ten years ago. But now I’m all grown up, if you know what I mean. So text me, any time, night or day. M’kay?

\chapter{Cycle 27}

\scene{Woody Balks}

\leadin{“Where are the parachutes?”} Woody Bumberschott asked nobody in particular.

One of the other two guys in the cabin of the plane (Woody couldn’t remember their names) shouted to the pilot (who had given the surely fictitious name Icarus Wallenda), “Hey Icky, this guy wants a parachute!”

Icky looked around at the guy who had spoken, then at Woody. Then he looked down at a clipboard that sat in the otherwise empty co-pilot seat. “What do you want a parachute for, Linwood?”

“Woody,” Woody said. “People call me Woody.”

“Why do they call you Woody,” Icky said, and looked at Woody’s crotch.

“Hey,” Woody said, reflexively covering his crotch with his hands.

The other two guys in the cabin laughed.

“So you don’t have parachutes?” Woody asked the pilot.

Icky shook his head.

Woody pointed to the co-pilot seat. “You mind if I sit here?”

“Be my guest,” Icky said. “We’ll be over the target soon, though.”

Woody sat. “You really don’t have parachutes?”

“What would be the point?” Icky said.

“I mean left over from before. When you did sky diving.”

“I didn’t need them then, either. People brought their own. You don’t want to trust something like that to some drunk assed has been pilot wannabe.”

“Wannabe? You’re not a real–”

“Relax,” said Icky. “I’m a real pilot.”

“Well, that’s reassuring.”

“And a real drunk, too, if you want to know the truth.”

Woody looked at the steering yoke in front of him. “Where’s your co-pilot?”

“Ain’t got no co-pilot,” Icky said. “What do you want a co-pilot for?”

“Well… What if something goes wrong?”

Icky laughed and turned to the two guys in the cabin. “Hey, Steve, what if something goes wrong?”

“What do you mean?” Steve said.

“Our friend Woody here is worried that something might go wrong!”

Steve and his buddy laughed.

Icky looked back at Woody. “Woody, my friend, what’s the worst that could happen?”

“Well…” Woody said, and trailed off. Dying wouldn’t be the worst thing that could happen. Dying was the point.

Icky said, “Well what the Christ did you think `skydying’ meant?”

“I don’t –”

“If something goes wrong, we die, which is the whole point. So, again, what could go wrong?”

“Well…” Woody said. “We could … I don’t know, crash or something but not die. Just be mangled up bad.”

“Well, one,” Icky said, holding up a finger, “by the time we get close enough to the ground to crash, you’ll be long gone. You’ll be part of the ground. And two, well, why does that matter? In three minutes you jump. In three minutes forty one seconds, give or take, you hit the ground at about a hundred and twenty miles per hour. In three minutes forty one and a tenth seconds, you’re beyond caring. So what’s this all about? You chickening out?”

“I don’t…” Woody said. His mouth felt dry.

Icky turned. “Vick, how many times you done this?”

Vick said, “This will make thirteen. Lucky thirteen.”

“How about you, Steve?”

“This makes ten,” Steve said. “I have to prove I’m not a cat, you know.”

“Anything ever go wrong? For either of you guys?”

“No,” Vick said. “Not unless you count total catastrophic biological malfunction, That’s happened a few times. Like, twelve maybe.”

Woody said, “What’s it like?”

Steve said, “It’s a total frigging rush. In every sense of the word. Biggest frigging panic of your life. Then you hit the ground and you’re back wherever you were on Friday afternoon.”

“Speaking of which,” Icky said, looking out left window of the cockpit, “we’re here. Anytime you’re ready…”

Steve stood up and leaned out the open portal, restraining himself with his hands on either side. The wind whipped his short blond hair. He pushed himself back into the plane, turned around and said to Vick, “See you next cycle?”

Vick nodded. Steve leaned back and fell out of the plane.

Woody turned to look out his cockpit window, but couldn’t see Steve’s fall. He thought he heard a scream, but wasn’t sure.

“Hey, Mister,” Vick yelled. “You coming?”

“In a minute,” Woody said. “Enjoy your trip.”

“My what?” Vick said. Then he winked, stepped forward, and tripped on a bump in the rubber mat that ran between the eight seats of the small cabin. “Oh!” he said as he fell out the portal.

Woody suspected that the “trip” had been a fake one.

“Your turn, sunshine,” said Icky.

Woody looked at Icky and blinked.

“You might as well,” Icky said. “I ain’t giving your money back either way. As if money made a gnat’s worth of difference anyway.”

Woody said, “Did you ever have anybody change their mind?”

“Not yet, I ain’t. What the Christ did you come up here for if you didn’t want to experience the ultimate frigging thrill ride?”

“I wanted to. I thought I wanted to.”

“And now you don’t?”

“I’m not sure.”

“I take great pride in giving people their money’s worth.”

“Even if they change their minds?”

“Well, let me rephrase that. I take great pride in giving people everything they paid for.”

“But…” Woody’s mouth was even dryer now. “What if they don’t want it any more?”

“Well,” Icky said. “Remember when I said it don’t make sense to worry about nothing going wrong?”

“Your evocative phrase, if I remember right, was something like ‘because you’ll be part of the ground.’”

“Well, that was number one,” Icky said. “I never did tell you number two, the other reason not to worry.”

“What’s… What’s the other reason?”

“Because if anything starts to go wrong, I can always just do this.” He dipped the steering yoke sharply toward his lap, and the plane tilted nose first toward the earth.
